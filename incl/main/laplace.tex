\chapter{Laplace}
\section{Introduction}
Up until this point, the paper has focused on the behaviour of electrical circuits in the time domain. Observing how different signals change with time yielded a differential equation, as seen in section (\ref{sec371}). \\ \\
The solution to a differential equation will often be difficult to derive, and this is where the Laplace transform comes in handy. The Laplace transform converts functions of time to functions of frequency (i.e. from the time domain to the \textit{s}-domain). This process reduces the differential equation in question to an algebraic equation. Once the expression has been solved in the \textit{s}-domain, the inverse Laplace transform can be applied to find its corresponding solution in the time domain.
\section{The Laplace Transform}
\begin{tcolorbox}[colback=blue!5!white,colframe=blue!75!black,title=Definition]
\begin{align}
\mathcal{L}\{f(t)\}=F(s)=\int_{0}^{\infty} e^{-st}\cdot f(t)\ dt
\label{lpdef}
\end{align}
where $f(t)$ is a function of time, $F(s)$ is the Laplace transform of that very function, and $s$ is a complex variable.
\end{tcolorbox}

\begin{tcolorbox}[colback=red!5!white,colframe=red!55!black,title=Example of a Laplace transform]
In (\ref{lpdef}), let $f(t)=e^{at}$, where $a$ is a constant, and $t$ is time. In that case,
$$\mathcal{L}\{f(t)\}=\int_{0}^{\infty} e^{-st}\cdot e^{at}\ dt$$
Since they have the base, $e$, in common, their exponents can be combined. Additionally, $t$ can be factorised:
$$\mathcal{L}\{f(t)\}=\int_{0}^{\infty} e^{-(s-a)t}\ dt$$
If $u=-(s-a)t$, then integration by substitution yields $dt=-\dfrac{1}{(s-a)}\ du$. Thus:
\begin{align}
\mathcal{L}\{f(t)\}=\int_{0}^{\infty} e^{u}\cdot (-\dfrac{1}{(s-a)})\ du =  -\dfrac{1}{s-a} \cdot \left[e^{-(s-a)t} \right]_{0}^{\infty}
\label{eq6.2}
\end{align}
Applying the limits of integration to (\ref{eq6.2}), and letting N approach $\infty$. Furthermore, the The equation is going to look as follows:
\begin{align*}
\mathcal{L}\{f(t)\} = \lim_{N \to \infty} -\dfrac{1}{s-a} \cdot \left[e^{-(s-a)t} \right]_{0}^{N} =-\dfrac{1}{s-a}\cdot (e^{-(s-a)N}-e^{0})=\dfrac{1}{s-a}
\end{align*}
In conclusion, the Laplace transform of $f(t)=e^{at}$ equals
$$\mathcal{L}\{f(t)\}=\dfrac{1}{s-a} \ \ \ ;\ \ \ s>a$$
$s$ must be greater than $a$, since the limit of $e^{-(s-a)N}$ converges towards zero. If $a$ is greater than $s$, the exponent of $e$ would be positive, and the limit would diverge. 
\end{tcolorbox}
The Laplace transform can be used for all functions of $f$, where 
The Laplace transform can also be applied to the derivatives of multiple orders. In particular, the Laplace transform for a first order derivative will be used when talking about high and low-pass filters. 
%mere tekst/bedre overgang
\begin{tcolorbox}[colback=green!5!white,colframe=green!40!black,title=Theorem 6.1: Laplace transform of a first order derivative]
If
$$\mathcal{L}\{f(t)\}=\int_{0}^{\infty} e^{-st}\cdot f(t)\ dt$$
then
$$\mathcal{L}\{f'(t)\} = s\cdot F(s)-f(0)$$
\end{tcolorbox}

\begin{tcolorbox}[colback=gray!5!white,colframe=gray!!black,title=Proof 6.1]
Integration by parts states that 
\begin{align}
\int_{a}^{b}{u\cdot v}=\left[u\cdot v\right]_{a}^{b}-\int_{a}^{b}u'\cdot v\
\label{eq6.3}
\end{align}

In this case, if $u=e^{-st}$ and $f'(t)=dv$, then $u'=du=-s\cdot e^{-st}$ and $v=f(t)$.
Inserting this in (\ref{eq6.3}) yields the following expression.
$$\mathcal{L}\{f'(t)\}=\left[e^{-st}\cdot f(t)\right]_{0}^{\infty}-\int_{0}^{\infty} -s\cdot e^{-st}\cdot f(t)\ dt$$

To rewrite, use that $\dfrac{1}{a}=a^{-n}$. Additionally, since $s$ is merely a constant, it can be placed in front of the integral. When applying the limits of integration to the left hand side, let $N$ approach infinity, such that
\begin{align}
\mathcal{L}\{f'(t)\}=\lim_{N \to \infty}\left[\dfrac{1}{e^{st}}\cdot f(t)\right]_{0}^{N}+s\cdot \int_{0}^{\infty}e^{-st}\cdot f(t)\ dt
\label{eq6.4}
\end{align}

Note, that the right hand side (???) of (\ref{eq6.4}) now equals $\mathcal{L}\{f(t)\}$. 
$$\mathcal{L}\{f'(t)\} = \lim_{N \to \infty}\left[\dfrac{1}{e^{s\cdot N}}\cdot f(N)-\dfrac{1}{e^{s\cdot 0}}\cdot f(0)\right]+s\cdot \mathcal{L}\{f(t)\}$$

Since $$\lim_{N \to \infty}\left[\dfrac{1}{e^{s\cdot N}}\right]=0$$ then, naturally, any other number multiplied by 0 will also equal to 0.\\
In conclusion, this results to
\begin{align*}
\mathcal{L}\{f'(t)\} = s\cdot \mathcal{L}\{f(t)\}-f(0)
\end{align*}
\end{tcolorbox}