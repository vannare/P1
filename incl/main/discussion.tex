\chapter{Discussion}
In chapter \ref{chap:RC}, an experiment with an RC circuit was made, where the capacitor was charged and discharged. The data from the experiment was compared with theoretical values. Despite the deviation the theoretical and raw data fits almost perfectly, with a coefficient of determination, $R^2$, close to $1$. The data is measured over 20 $\tau$, which is equal to $9.32 ms$ according to \eqref{rc:tau:full} in chapter \ref{chap:RC}. Considering the accuracy of the measuring instrument, then it is incredibly precise down to a few milliseconds. 
\\ \\
In chapter \ref{chap:high_low}, a low- and high-pass filter was constructed. For the low-pass filter the voltage was measured across the capacitor, and across the resistor for the high-pass filter. From the data the cut-off frequency for the low-pass filter was found to be $2112.22 \frac{rad}{s}$ and $2153.69 \frac{rad}{s}$ for the high-pass filter. Whereas the theoretical cut-off frequency was $2148 \frac{rad}{s}$. For the low- and high-pass filter there was a deviation of $1.67 \% $ and $0.26 \% $ respectively. Furthermore, the data for the bode plot for both the high- and low-pass filters are almost identical to the simulated values. The cut-off frequency is found as the data point closest to the simulated cut-off frequency value. This is biggest source of error connect with the cut-off frequency. Since there are only 1001 data-points in the simulated data, there will naturally be a small difference between the simulated data and the experiment, even though the lines align perfectly. The biggest source of error connected with the bode plot is the measuring of the lower decibels. The random noise in the measuring becomes relatively bigger, as the decibels decreases and it therefore becomes visible on the bode plot.