\chapter{Differential Equations} 
The theory in this chapter is based on \cite{diffandcomplex}.\\ In this chapter linear differential equations of the first order, separable differential equations, and their solutions will be described with corresponding examples. \\

%Forslag: 
% The following chapter introduces and describes linear differential questions of the first order, separable differential equations, and their respective solutions, as well as corresponding examples. The chapter is based on (cite), and uses the Leibniz notation for derivatives.
\noindent Before describing an RC circuit mathematically, the concept of differential equations has to be described. Differential equations are important tools used to describe dynamic systems. An example of this could be an equation describing an increasing population. The growth of a population is usually a function of that same population. This means that it is easier to describe changes in the population, rather than the population as a whole. \\

% Forslag:
%The concept of differential equations arises when describing RC circuits mathematically. Differential equations are important tools in describing dynamic, physical systems. An example of such an equation could describe an increasing population. The growth of a function is usually a function of that very population.

\begin{definition}{Differential equation}{}
A differential equation is an equation that contains a function, and one or more of its derivatives.
\end{definition}
\noindent%Phenomena
These equations have a class of functions as a solution, rather than a number. 
\\
A differential equation could look like this: $\dfrac{dy}{dx} = y$. This equation asks: "What function is its own derivative?", which is in this case: $y=e^x+C$.
\\\\
In this context, $dy$ and $dx$ can be perceived as small changes in $y$, and $x$. The slight change in $y$ with respect to $x$ can then be described as $\dfrac{dy}{dx}$.
\\
This is formally expressed as:
\\
\begin{align*}
	\dfrac{dy}{dx} =& \lim_{x_0\to 0} \dfrac{y(x+x_0)-y(x)}{x_0}.
\end{align*}

\noindent The solution to a differential equation can either be general or particular. A general solution is a class of functions, that differs by a constant.
\\
To find a particular solution to a differential equation an initial condition has to be given. Meaning that if a point on the curve is known, a particular solution from the general solutions can be found.

\clearpage

\begin{definition}{Order of a differential equation}{}
The order of a differential equation is determined by the highest order of derivative present in the equation.
\end{definition} 

\noindent
\textbf{Examples of differential equations of different orders:}
\\
A first order differential equation:
$$\frac{dy}{dx}-4y=0. $$
A second order differential equation:
$$\frac{d^2y}{dx^2}+\frac{dy}{dx}+y = 0.$$
And a third order differential equation:
$$\frac{d^3y}{dx^3} - 9\frac{d^2y}{dx^2} + 15\frac{dy}{dx} + 25y = 0.$$

\section{Linear differential equations}
A differential equation can either be linear or non-linear. A linear differential equation is a linear polynomial that consists of a function and its derivatives.
\begin{definition}{$N$'th order linear differential equation}{}
A linear $N$'th order differential equation is an equation that takes the form:
\begin{align*}
\sum_{k=0}^{N}\left(\dfrac{d^ky}{dx^k}a_k(x)\right)+b(x)=0,
\end{align*}
where $a_k(x)$ are continuous functions on a given interval.
\end{definition}
\subsection{Solving a linear differential equation of the first order}
Solving  a differential equation is not always easy or possible, however, for differential equations in certain forms, a general solution can be found.

\begin{theorem}{General solution to a linear differential equation of the first order}{linethe}
A first order linear differential equation in the form of:
\begin{align} \label{FODE_form}
\dfrac{dy}{dx}+h(x)y=g(x),
\end{align}
has the general solution:
\begin{align} \label{FODE_solution}
y=e^{-H(x)}\left(\int e^{H(x)}g(x)\ dx+C\right),
\end{align}
where $H(x)$ is the antiderivative of $h(x)$, $h$ and $g$ are continuous on a given interval, and $C\in \mathbb{R}$.
\end{theorem}

\begin{prof}{}{}
\Cref{FODE_form} is multiplied by $e^{H(x)}$ on each side:
\begin{align*}
\left(\dfrac{dy}{dx}+h(x)y\right)e^{H(x)}=& g(x)e^{H(x)}.
\end{align*}
Now $e^{H(x)}$ is multiplied into the parentheses:
\begin{align*}
\dfrac{dy}{dx}e^{H(x)}+h(x)ye^{H(x)}=& g(x)e^{H(x)}.
\end{align*}
In this case, due to the chain rule, $h(x)ye^{H(x)} = y \cdot \dfrac{d}{dx} \left(e^{H(x)} \right)$:
\begin{align*}
\dfrac{dy}{dx} \cdot e^{H(x)} + y \cdot \dfrac{d}{dx}e^{H(x)}=& g(x)e^{H(x)}.
\end{align*}
Using the chain rule the equation can then be rewritten:
\begin{align*}
\dfrac{d}{dx}\left(e^{H(x)}y\right)=g(x)e^{H(x)}.
\end{align*}
Both sides are then integrated with respect to $x$:
\begin{align*}
\int\dfrac{d}{dx}\left(e^{H(x)}y\right)\ dx=&\int g(x)e^{H(x)}\ dx,
\\
e^{H(x)}y=&\int g(x)e^{H(x)}\ dx+C,
\end{align*}
where $C$ is the constant of integration from the left-hand side of the  equation. By multiplying both sides with $e^{-H(x)}$, the general solution can be found:
\begin{align*}
y=e^{-H(x)}\left(\int g(x)e^{H(x)}\ dx+C\right).
\end{align*}
\end{prof}

\begin{example}{Solving a linear differential equation of first order}{}
Consider the following differential equation with the given initial condition:
\begin{align}
	\dfrac{dy}{dx}-2y = e^x, \label{ODE_ex} ~~~~y(0) = 5.
\end{align}
Equation \eqref{ODE_ex} is in the form of \eqref{FODE_form}, where:
%This equation can be rearranged to fit the form of \eqref{FODE_form}, and a general solution, to the differential equation, can then be found using \eqref{FODE_solution}:
\begin{align*}
	h(x) =& -2, \\
	g(x) =& e^x, \\
	H(x) =& \int{-2 \ dx}, \\
	     =& -2x + C.
\end{align*}
Substituting these terms into \eqref{FODE_solution} yields: 

\begin{align*}
	y(x)=&e^{2x-C}\left(\int{e^{-2x+C}e^x\ dx}+C\right), \\
	y(x)=&e^{2x}e^{-C}\left(e^{C}\int{e^{-2x}e^{x}\ dx}+C\right), \\
	y(x)=&e^{2x}\left(\int{e^{-x}\ dx}+C_{1}\right),
\end{align*}
where $C_{1}$ is the new constant of integration, $C_1=e^{-C}C$. \\
The integral of $e^{-x}$ can then be found. The following result is then reduced:
\begin{align*}
	y(x)=&e^{2x}\left(-e^{-x}+C_{2}+C_{1}\right), \\
	y(x)=&-e^x+C_{3} \cdot e^{2x},
\end{align*}
where $C_{3}$ is the new constant of integration, $C_{3}=C_{2}+C_{1}$. The constant is then found from the initial condition:
\begin{align*}
	y(0)=& ~ 5, \\
	5=&-e^0+C_{3} \cdot e^{2 \cdot 0}, \\
	5 =& ~ C_{3}-1, \\
	C_{3} =& ~ 6.
\end{align*}
The particular solution to the differential equation is then:
\begin{align*}
	y(x) = -e^x+6e^{2x}.
\end{align*}
To check the validity of this solution, it is substituted into the differential equation:
\begin{align*}
	e^x =& \dfrac{dy}{dx} -2y, \\
	e^x =& -e^x+12e^{2x} -2\left(-e^x+6e^{2x}\right),  \\
	e^x =& -e^x+2e^x,  \\
	e^x =& e^x.
\end{align*}
This is the particular solution to the given differential equation.
\end{example}

\section{Separable differential equations}\label{SepDiff}
\begin{definition}{Separable differential equations}{def:SDE}
Given a differential equation in the following form: 
$$\frac{dy}{dx} = f(x,y).$$
The equation is said to be separable if the right-hand side of the equation can be expressed as the product of two functions, $g(x)$ and $p(y)$, depending only on one variable each, such that:
\begin{align*}
f(x,y)=g(x)p(y),
\end{align*}
where $g(x)$ and $p(y)$ are continuous on a given interval.
\end{definition}
\noindent
\begin{theorem}{Solving a separable differential equation}{}
Given a separable differential equation in the form:
\begin{align}
\dfrac{dy}{dx}=g(x)p(y), \label{SDEG}
\end{align}
it can be solved as follows:
\begin{align*}
\int h(y)\ dy =&\int g(x)\ dx.
\end{align*}
Where $h(y)=\dfrac{1}{p(y)}$ and $g, p$ and $h$ are continuous functions on a given interval.
\end{theorem}
\begin{justification}{}{}
Let $h(y) = \dfrac{1}{p(y)}$:
 \begin{align}
	\frac{dy}{dx} &= g(x)p(y), \nonumber\\
	h(y)\frac{dy}{dx} &= g(x). \label{eq:1}
 \end{align}
Then by letting $H(y)$ be the antiderivative of $h(y)$, and $G(x)$ be the antiderivative of $g(x)$, \eqref{eq:1} can be rewritten as: 
 \begin{align}
 	\frac{dH}{dy}\frac{dy}{dx} &= \frac{dG}{dx}. \label{eq:2}
 \end{align}
Using the chain rule for differentiation, the left-hand side of the equation can be written as the derivative of the composite function $H \left(y(x) \right)$: $\dfrac{dH}{dy}\dfrac{dy}{dx} =\dfrac{d}{dx} H \left(y(x) \right).$ This can be substituted in \eqref{eq:2}, such that:
 \begin{align}
 	\frac{d}{dx}H\left(y(x)\right) &= \frac{dG}{dx}\label{eq:4}.
 \end{align}
Equation \eqref{eq:4} shows that $H\left(y(x)\right)$ and $G(x)$ have the same derivative, and must differ by a constant $C$ after integrating both sides:
 \begin{align*}
 	H\left(y(x)\right) &= G(x) + C.
 \end{align*}
This shows that the same result is found whether $\dfrac{dy}{dx}$ is split algebraically, or the antiderivatives are used together with the chain rule for differentiation. Therefore, it is justified to treat $\dfrac{dy}{dx}$ algebraically.
\end{justification}

\begin{example}{Solving a separable differential equation}{}
Consider the differential equation with the initial condition:
\begin{align*}
	\dfrac{dy}{dx} = y^2 4x, ~~~
	y(1) = \dfrac{1}{3},
\end{align*}
The terms, in this differential equation are separable: 
\begin{align*}
	\dfrac{dy}{dx} y^{-2} =& 4x, \\
	y^{-2} dy =& 4x dx. 
\end{align*}
Both sides of the equation are then integrated, and solved for $y$:
\begin{align*}
	\int{y^{-2}\ dy} =& \int{4x\  dx}, \\
	-y^{-1} =& 2x^2 +C, \\
	y =& \dfrac{1}{-2x^2-C}.
\end{align*}
Using the initial condition, the constant of integration can then be found:
\begin{align*}
	y(1) =& \dfrac{1}{3}, \\
	\dfrac{1}{3} =& \dfrac{1}{-2 \cdot 1^2-C}, \\
	3 =& -2-C, \\
	C =& -5.
\end{align*}
The particular solution to this equation is then:
\begin{align*}
	y(x) = \dfrac{1}{-2x^2+5}.
\end{align*}

The validity of the solution can be checked by substituting the solution in the equation:

% "equation" mangler reference til et udtryk

\begin{align*}
	\dfrac{dy}{dx} = y^24x, ~~~
	y(1) = \dfrac{1}{3},
\end{align*}
\begin{align*}
	\dfrac{d}{dx} \left(\dfrac{1}{-2x^2+5}\right) =& \dfrac{4x}{(-2x^2+5)^2}, \\
	-4x\left(\dfrac{-1}{(-2x^2+5)^2}\right)  =& \dfrac{4x}{(-2x^2+5)^2}, \\
	\dfrac{4x}{(-2x^2+5)^2} =& \dfrac{4x}{(-2x^2+5)^2}.
\end{align*}
A particular solution to the equation has then been found.
\end{example}
