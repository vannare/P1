\chapter{Differential equations}

\section{What is a differential equation?}
A differential equation is any equation, which includes a derivative. These equations aren't solved with a number, like a traditional equation. These equations are solved with a function, or rather a class of function. \\
A 'normal' equation would look something like this: $2x+4=12$. This equation could be solved with an $x=4$. However, a differential equation would look something like this instead: $\frac{dy}{dx} = y$. This equation asks "What function is it's own derivative?", which we of course can answer with $y=e^x$. \\

\subsection{Some examples of a differential equation}
We know from physics that the force applied to an object, is the product of the object's mass, and the acceleration resulting from the applied force. This is often written as: 
\begin{align*}
	F=ma
\end{align*}
Where $F$ is force, $m$ is mass, and $a$ is acceleration. \\
We know that acceleration is a change in speed over some time interval. This means that acceleration is the derivative of speed, with respect to time:
\begin{align}
	a = \frac{dv}{dt}
\end{align}
it also follows that speed is a change in position, over time. Meaning that acceleration can be described with the following equation:
\begin{align*}
	a = \frac{dv}{dt} \rightarrow a = \frac{d^2u}{dt^2}
\end{align*}
Here $u$ represents the object position, meaning that $du$ is the objects change in position. \\
Thus force can be described as follows:
\begin{align}
	 F = m\frac{d^2u}{dt^2}
\end{align}
\\
The following are a few other examples of differential equations:
\begin{align}
	af''+bf'+cf =& 5\frac{dy}{dx}		\\
	24cos(t)=&\frac{d^3y}{dx^3}			\\
	f(x)=4f(x) 						
\end{align}

\subsection{The order of a differential equation}
The order of a differential equation, is denoted as the the degree of the largest degree derivative, present in the equation. Meaning that (2.3) is a second order differential equation. (2.4) is a third order differential equation, and (2.5) is a first order differential equation.

\subsection{Separable differential equations}
A separable differential equation is an equation that takes the form:
\begin{align}
	G(y)\frac{dy}{dx}=H(x)
\end{align}
The equation can then be re-written and solved as follows, assuming that the integral exists\footnote{Assume that $dx$ and $dy$ can be manipulated algebraically. }:
\begin{align}
	G(y)dy=&H(x)dx				\\
	\int G(y)dy =& \int H(x)dx
\end{align}
From here the functions would just need to be integrated, as usual.\\
From there y could be isolated, and the integration constant could be found, if there exists an initial condition for the differential equation.
\subsection{Example of separable function}
What would the solution be to the following differential equation, with the given condition?
\begin{align*}
	\frac{dy}{dx} =& 6y^2x\\
	y(1)=&\frac{1}{25}
\end{align*}
The first step is to rewrite the equation, such that it follows the form of (2.6):
\begin{align*}
	\frac{1}{y^2}\frac{dy}{dx}=6x
\end{align*}
Then rewrite the equation to fit the form of (2.7), and then integrate both sides the equations:
\begin{align*}
	\frac{1}{y^2}dy=&6x dx				\\
	\int \frac{1}{y^2}dy=&\int 6x dx		\\
	-\frac{1}{y}=&3x^2+c		
\end{align*}
Given the initial condition the integration constant can be found:
\begin{align*}
	-\frac{1}{\frac{1}{25}}=&3+c
	c=%-28
\end{align*}
This would mean that y could be isolated, and the equation would be solved in this specific case:
\begin{align*}
	-\frac{1}{y}=&3x^2-28\\
	y=&-\frac{1}{3x^2-28}
\end{align*}