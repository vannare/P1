\chapter{Differential equations}
\section{What is a differential equation?}
\begin{tcolorbox}[colback=blue!5!white,colframe=blue!75!black,title=Definition: Differential equation] 
A differential equation is an equation which, contains one or more derivatives of a function.
\end{tcolorbox}
These equations have a function as a solution, rather than a number. 
\\
A differential equation would look something like this: $\frac{dy}{dx} = y$. This equation asks: "What function is its own derivative?", which is of course $y=e^x+C$.
\\
A differential equation can either have a general solution or a particular solution. When finding the general solution, there is no initial condition, to determine the constant $C$. Finding a particular solution requires an initial condition, from which the constant $C$ can be found.
\\
\begin{tcolorbox}[colback=blue!5!white,colframe=blue!75!black,title=Definition: Order of a differential equation]
The order of a differential equation is determined by the highest derivative present in the differential equation.
\end{tcolorbox} 
\begin{tcolorbox}[colback=red!5!white,colframe=red!55!black,title=Examples of differential equations of different orders] 
A first order differential equation:
$$f(x)=4\frac{df}{dx} $$
A second order differential equation:
$$a\frac{d^2y}{dx^2}+b\frac{dy}{dx}+cy = 5\frac{dy}{dx}$$
And a third order differential equation:
$$\frac{d^3y}{dx^3} - 9\frac{d^2y}{dx^2} + 15\frac{dy}{dx} + 25y = 0$$
\end{tcolorbox}
\section{Linear differential equations}
Linear differential equations are differential equations in which a function and its derivatives appear only to the power of one or zero. The functions in a linear differential equation must not be multiplied by each other.
\\
Functions containing $\dfrac{d}{dx}y^2$, $\dfrac{dy}{dx}y$, or $e^y$ are non-linear. Mathematically, a function is said to be linear if it satisfies the following conditions:
\begin{align*}
\text{Additivity}:
f(x+y)=&f(x)+f(y)
\\
\text{Homogeneity}:
f(kx)=&kf(x)
\end{align*}
As an example: $\dfrac{dy}{dx}+y=0$ is linear, where $\dfrac{dy}{dx}+y^2=0$ is non-linear.
\begin{tcolorbox}[colback=blue!5!white,colframe=blue!75!black,title=Definition: $N$'th order linear differential equation]
A linear $N$'th order differential equation is an equation, which can be written in the following way:
\begin{align*}
\sum_{k=0}^{N}\left(\dfrac{d^ky}{dt^k}a_k(t)\right)+b(t)=0,
\end{align*}
where $a_k(t)$ are continuous functions on a given interval.  
\end{tcolorbox}
\subsection{Solving a linear differential equation of the first order}
\begin{tcolorbox}[colback=green!5!white,colframe=green!40!black,title=Theorem 2.1: General solution to a linear differential equation of the first order]
A first order linear differential equation in the form of:
\begin{align} \label{FODE_form}
\dfrac{dy}{dx}+h(x)y=g(x),
\end{align}
has the general solution:
\begin{align} \label{FODE_solution}
y=e^{-H(x)}\left(\int e^{H(x)}g(x)dx+C\right),
\end{align}
where $H(x)$ is the antiderivative of $h(x)$, $h$ and $g$ are continuous in a given interval, and $C\in \mathbb{R}$.
\end{tcolorbox}
\begin{tcolorbox}[colback=gray!5!white,colframe=gray!!black,title=Proof 2.1]
The equation \eqref{FODE_form} is multiplied by $e^{H(x)}$ on each side.
\begin{align*}
\left(\dfrac{dy}{dx}+h(x)y\right)e^{H(x)}=&g(x)e^{H(x)}
\\
\dfrac{dy}{dx}e^{H(x)}+h(x)ye^{H(x)}=&g(x)e^{H(x)}
\end{align*}
Due to the product rule when differentiating, the equation can be rewritten as:
\begin{align*}
\dfrac{d}{dx}\left(e^{H(x)}y\right)=g(x)e^{H(x)}
\end{align*}
Both sides are then integrated with respect to $x$:
\begin{align*}
\int\dfrac{d}{dx}\left(e^{H(x)}y\right)dx=&\int g(x)e^{H(x)}dx
\\
e^{H(x)}y=&\int g(x)e^{H(x)}dx+C,
\end{align*}
where $C$ is the constant of integration from the left side of the  equation. By multiplying both sides with $e^{-H(x)}$, the general solution can be found.
\begin{align}
y=e^{-H(x)}\left(\int e^{H(x)}g(x)dx+C\right)
\end{align}
\end{tcolorbox}

\section{Separable differential equations} \label{SDE}
\begin{tcolorbox}[colback=blue!5!white,colframe=blue!75!black,title=Definition: Separable equation] 
    $$\frac{dy}{dx} = f(x,y)$$
    If the right-hand side of the equation can be expressed as   a function $g(x)$ that depends only on $x$, times a function $p(y)$ that depends only on $y$, then the differential equation is called separable.
   \end{tcolorbox}
As per definition, a first order differential equation can be written as:
\begin{align}
	\frac{dy}{dx}=g(x)p(y)
	\label{2_4}
\end{align}
The equation can then be rewritten and solved as follows:
\begin{align}
	h(y)dy=g(x)dx,
\end{align}
where $h(y) = \frac{1}{p(y)}$. Both sides of the equation are then integrated:
 \begin{align}
 	\int h(y)dy =&\int g(x)dx   \\
 	H(y)=&G(x)+C
 \end{align}
 
The two constants are merged into one constant $C$. This is the general solution to a differential equation. \citep{diffandcomplex}

\subsection{Justification} 
To show that the differentials $dy$ and $dx$ in $\frac{dy}{dx}$ can be treated algebraically when solving \eqref{2_4}:
 \begin{align}
	\frac{dy}{dx} &= g(x)p(y)\nonumber\\
	h(y)\frac{dy}{dx} &= g(x)\label{eq:1}
 \end{align}
Then by letting $H(y)$ be the antiderivative of $h(y)$, and $G(x)$ be the antiderivative of $g(x)$, equation \eqref{eq:1} can be rewritten as: 
 \begin{align}
 	\frac{dH}{dy}\frac{dy}{dx} &= \frac{dG}{dx}\label{eq:2}
 \end{align}
Using the chain rule for differentiation, $\frac{dH}{dy}\frac{dy}{dx}$ can be written as the derivative of the composite function $H(y(x))$:
 \begin{align*}
	\frac{d}{dx} H(y(x)) &= \frac{dH}{dy}\frac{dy}{dx}.
 \end{align*}
In \ref{eq:2} this term can be substituted:
 \begin{align}
 	\frac{d}{dx}H(y(x)) &= \frac{dG}{dx}\label{eq:3}.
 \end{align}
Equation \ref{eq:3} shows that $H(y(x))$ and $G(x)$ have the same derivative, and must differ by a constant after integrating both sides:
 \begin{align*}
 	H(y(x)) &= G(x) + C
 \end{align*}
This shows that the same result is found whether $\frac{dy}{dx}$ is split algebraically, or the antiderivatives are used together with the chain rule for differentiation. Therefore, it is justified to treat $\frac{dy}{dx}$ algebraically.

\begin{tcolorbox}[colback=red!5!white,colframe=red!55!black,title=Example of solving a separable differential equation] 
What would the solution be to the following differential equation, with the given initial condition?
\begin{align*}
	\frac{dy}{dx} =& 6y^2x\\
	y(1)=&\frac{1}{25}
\end{align*}
The first step is to rewrite the equation, such that it follows the form of \eqref{eq:1}
\begin{align*}
	\frac{1}{y^2}\frac{dy}{dx}=6x
\end{align*}
Then rewrite the equation to fit the form of (2.7), and then integrate both sides the equations:
\begin{align*}
	\frac{1}{y^2}dy=&6x dx				\\
	\int \frac{1}{y^2}dy=&\int 6x dx		\\
	-\frac{1}{y}=&3x^2+C		
\end{align*}
Given the initial condition the integration constant can be found:
\begin{align*}
	-\frac{1}{\frac{1}{25}}=&3\cdot 1^2+C	\\
	C=&-28
\end{align*}
This would mean that $y$ could be isolated, and the equation would be solved in this specific case:
\begin{align*}
	-\frac{1}{y}=&3x^2-28\\
	y=&-\frac{1}{3x^2-28}
\end{align*}
\end{tcolorbox}

