\chapter{Differential equations}
\section{What is a differential equation?}
A differential equation is any equation, which includes a function and its derivative. These equations have function as the solution, rather than a number.  \\
The general form of an differential equation look like this
\begin{align*}
efef
\end{align*}

A differential equation can either have general solution or a complete solution. When finding the general solution, there is no starting in which the constant $c$ can be found. The complete solution, there is a given starting point and the constant $c$ can be found, and there is not any unknown parts of the solution. \\

A differential equation would look something like this: $\frac{dy}{dx} = y$. This equation asks "What function is its own derivative?", which, is of course $y=e^x$. \\




\begin{tcolorbox}[colback=red!5!white,colframe=red!75!black,title=Example of differential equation of different orders] 
A first order differential equation:
$$f(x)=4\frac{df}{dx} $$
A second order differential equation:
$$a\frac{d^2y}{dx^2}+b\frac{dy}{dx}+cy = 5\frac{dy}{dx}$$
And a third order differential equation:
$$\frac{d^3y}{dx^3} - 9\frac{d^2y}{dx} + 15\frac{dy}{dx} + 25y = 0$$
\end{tcolorbox}
\section{Linear differential equations}
A linear $N$'th order differential equation are equations which can be written in following way.
\begin{align*}
\sum_{k=n}^{N}a_k\dfrac{d^k}{{dx}^k}y = 
\end{align*}
\\
\textbf{Solving linear FODE}
\\
A first order linear differential equation in the form of:
\begin{align} \label{FODE}
\dfrac{dy}{dx}+h(x)y=g(x)
\end{align}
Has the general solution:
\begin{align}
y=e^{H(x)}\left(\int e^{H(x)}g(x)dx+C\right)
\end{align}
Where $H(x)$ is the anti derivative of $h(x)$ and $h$ and $g$ are continuous in a given interval.
\\
\textbf{Proof}
\\
Starting with \eqref{FODE}, the equation is multiplied by $e^{H(x)}$ on each side.
\begin{align*}
\left(\dfrac{dy}{dx}+h(x)y\right)e^{H(x)}=g(x)e^{H(x)}
\\
\dfrac{dy}{dx}e^{H(x)}+h(x)ye^{H(x)}=g(x)e^{H(x)}
\end{align*}
Due to the product rule when differentiating, the left side of the equation can be rewritten as.
\begin{align*}
\dfrac{d}{dx}\left(e^{H(x)}y\right)=g(x)e^{H(x)}
\end{align*}
Bother side at then integrated with respect to x.
\begin{align*}
\int\dfrac{d}{dx}\left(e^{H(x)}y\right)dx=\int g(x)e^{H(x)}dx
\\
e^{H(x)}y=\int g(x)e^{H(x)}dx+C
\end{align*}
Where $C$ is the constant of integration from the left side of the  equation. By multiplying both sides with $e^{-H(x)}$ the general solution can be found.
\begin{align}
y=e^{-H(x)}\left(\int e^{H(x)}g(x)dx+C\right)
\end{align}
\subsection{Separable differential equations} \label{SDE}
\begin{tcolorbox}[colback=blue!5!white,colframe=blue!75!black,title=Definition: Separable equation] 
    $$\frac{dy}{dx} = f(x,y)$$
    if the right-hand side of the equation can be expressed as   a function $h(x)$ that depends only on $x$ times a function $g(y)$ that depends only on $y$, then the differential equation is called separable.
   \end{tcolorbox}
As per this definition, a first order differential equation can be written as
\begin{align}
	\frac{dy}{dx}=h(x)g(y)
\end{align}
The equation can then be re-written and solved as follows, assuming that the integral exists
\begin{align}
	g(y)dy=h(x)dx
\end{align}
Then both side is integrated
 \begin{align}
 \int g(y)dy =\int h(x)dx   \\
 H(y)=G(x)+C
 \end{align}
 
Then the two constants is merged into one symbol $C$, this is the general solution to a differential equation. \citep{diffandcomplex}

\begin{tcolorbox}[colback=red!5!white,colframe=red!75!black,title=Example using separable differential equation] 
What would the solution be to the following differential equation, with the given condition?
\begin{align*}
	\frac{dy}{dx} = 6y^2x, y(1)=\frac{1}{25}
\end{align*}
The first step is to rewrite the equation, such that it follows the form of (2.6):
\begin{align*}
	\frac{1}{y^2}\frac{dy}{dx}=6x
\end{align*}
Then rewrite the equation to fit the form of (2.7), and then integrate both sides the equations:
\begin{align*}
	\frac{1}{y^2}dy=&6x dx				\\
	\int \frac{1}{y^2}dy=&\int 6x dx		\\
	-\frac{1}{y}=&3x^2+c		
\end{align*}
Given the initial condition the integration constant can be found:
\begin{align*}
	-\frac{1}{\frac{1}{25}}=&3\cdot 1^2+c	\\
	c=&-28
\end{align*}
This would mean that y could be isolated, and the equation would be solved in this specific case:
\begin{align*}
	-\frac{1}{y}=&3x^2-28\\
	y=&-\frac{1}{3x^2-28}
\end{align*}
\end{tcolorbox}

\subsection{Justification} 
To show that the differentials $dy$ and $dx$ in $\frac{dy}{dx}$ can be treated algebraically when solving 
 \begin{align}
	\frac{dy}{dx} &= g(x)p(y)\nonumber\\
	h(y)\frac{dy}{dx} &= g(x)\label{eq:1},
 \end{align}
then by letting $H(y)$ be the antiderivative of $h(y)$ and $G(x)$ be the antiderivative of $g(x)$, equation \ref{eq:1} can be rewritten as 
 \begin{align}
 	H'(y)\frac{dy}{dx} &= G'(x)\label{eq:2}.
 \end{align}
With the help of the chain rule, $H'(y)\frac{dy}{dx}$ is the derivative of the composite function $\frac{d}{dx}H(y(x))$:
 \begin{align*}
	\frac{d}{dx} H(y(x)) &= H'(y(x))\frac{dy}{dx}.
 \end{align*}
This term can be substituted in in equation \ref{eq:2}:
 \begin{align}
 	\frac{d}{dx}H(y(x)) &= G'(x)\label{eq:3}.
 \end{align}
Equation \ref{eq:3} shows that $H(y(x))$ and $G(x)$ have the same derivative, and must differ by a constant:
 \begin{align*}
 	H(y(x)) &= G(x) + C
 \end{align*}
This shows that the same result is found wether $\frac{dy}{dx}$ is split algebraically or the antiderivatives are used together with the chain rule. Therefor it is justified to treat $\frac{dy}{dx}$ algebraically.