\chapter{Basic circuit theory}
Electrical circuits are fundamental building blocks in, more or less, every electronic device you can think of. The simple act of flipping a switch, e.g. to turn on the light, completes an electrical circuit. The purpose of the circuit is to carry electrical current, either in an open or closed circuit. The electrical components in a circuit are typically resistors, capacitors,  switches, and an electrical 	source (a battery, for instance).
\\ 
In this chapter, if a function is assumed constant, it is denoted with capitalization of its original notation. 
\\ 
\\
The different elements of the circuit can be divided into two categories: active and passive. Active elements supply energy to the system, e.g. a battery, whereas passive elements absorb energy, e.g. a light bulb. A circuit with a battery and a bulb can be represented in the following way.
\begin{figure}[H]
\begin{center}
\begin{circuitikz}[american voltages]
\draw
to[battery, battery1=$V_{B}$, color=blue] (0,2)

to[short, -] (2,2)
[short](2,2)

[short] (2,2)
to [lamp, l=$R_{\text{bulb}}$, color=green](2,0)

(0,0) to [short] (2,0);
\end{circuitikz}
\end{center}
\caption{A circuit with a battery and a light bulb}
\label{fig:bulb}
\end{figure} 
In figure \ref{fig:bulb} the active element is a battery, and the passive element is a light bulb. Charge is transported by the current from the positive terminal of the battery, through to the light bulb. The light bulb absorbs the energy from the charge, which is then transported to the negative terminal of the battery.
\\
\section{Current}
Current is the force that moves charge through a circuit. Current can be defined as an amount of charge moved over a time interval. This can be expressed as the following relation:
\begin{align}
i(t)=\dfrac{dq(t)}{dt} \Leftrightarrow q(t)=\int_{0}^{t}i(x)dx,
\end{align}
where $i(t)$ is the current (in ampere, $A$), to a given time $t$ (in seconds, $s$), and $q(t)$ is the function for charge at a given time $t$. $q(t)$ is measured in Coulomb$(C)$.
\\
There exists two types of current, alternating current (AC) and direct current (DC). DC current is constant, while AC alternates, see figure \ref{fig:ACDC}. 
\begin{figure}[H] 
\begin{tikzpicture}
\begin{axis}[ticks=none,
axis lines =center,
xlabel={t},
ylabel={i(t)},
    height=7cm, width=9cm,
    xmin=0, xmax=10, ymin=-2, ymax=2]
\addplot [
    domain=0:10, 
    samples=100, 
    color=red,
]
{1};
\addlegendentry{$DC$}
\addplot [
    domain=0:10, 
    samples=100, 
    color=blue,
    ]
    {sin(\x r)};
\addlegendentry{$AC$}
\end{axis}
\end{tikzpicture}
\caption{Current for AC and DC versus time}
\label{fig:ACDC}
\end{figure}
\section{Voltage}
Voltage ($V$), also called electric potential difference, is the change in potential energy a charge undergoes, when it passes through two given points in a circuit. This is expressed in the following equation:
\begin{align}
	V=\dfrac{dU(q)}{dq},
\end{align}
\\
where $U(q)$ (in joules, $J$) is the function for potential energy, given a charge $q$.
\section{Resistor}
When a resistor, which is a passive element, is added to the circuit, it creates a resistance ($\Omega$). Resistance makes it more difficult for the current to pass through the element. Resistance is defined as the proportional constant between current and voltage. The mathematical relation of this is given by:
\begin{align} 
\label{Ohm}
v(t)=R\cdot i(t),  R\geq0,
\end{align}
where $R$ is resistance (in Ohm, $\Omega$).
\section{Capacitor}
A capacitor is a passive element of a circuit. A capacitor consists of two similar sized plates. When a voltage is applied to the circuit, the capacitor gets charged. The capacitance is the amount of energy a capacitor can store, when it is fully charged. The capacitor gets charged when positive a charge is transferred from one plate, through the circuit, to the other plate. The capacitance is given by the following equation:
\begin{align*}
C=\dfrac{\epsilon_{0}A}{d},
\end{align*}
where $C$ is the capacitance (in farad, $F$), $\epsilon_{0}$ is the permittivity of free space, which is equal to $8.85 \cdot 10^{-12}                                                 \frac{F}{m}$. $A$ is the surface area of the plates (in square meters, $m^{2}$), and $d$ is the distance between the two plates (in meters, $m$).
\\
The charge of a capacitor across a voltage ($V$) and capacitance of ($C$) is equal to:
\begin{align}
\label{QCV}
Q=CV	
\end{align}
\section{Circuit diagrams}
Electrical circuits are visually represented in circuit diagrams. In addition to the above-mentioned elements, the circuit diagrams introduce three terms: nodes, branches, and loops. Elements are \textit{branches} i.e.  the voltage supply, resistors, capacitors, and the like. \textit{Nodes} connect the \textit{branches} of the circuit. Lastly, any closed path in the circuit, in which no node is encountered more than than once, is called a \textit{loop} \cite[page~32]{bcircuit}. An example of a circuit is shown below.

\begin{figure}[H]
 \begin{center}
\begin{circuitikz}[american voltages]
\draw
to[battery, battery1=$V_{B}$, color=blue] (0,2)
to[resistor, R=$R_1$, color=red] (2,2)
to[resistor, R=$R_2$, color=red] (2,0)
to[short, -] (0,0)
[short, -](2,2) to [short, -] (3,2)
to[resistor, R=$R_3$, color=red](3,0)
to[short, -] (2,0)
[short](3,2) to [short] (5,2)
to [C=$C$, color=green](5,0)
to [short, -] (3,0)
(0,0) to [short, l_=$N_3$, -] (5,0)
(2,2) to [short, l^=$N_2$, -] (5,2)
(0,1) to [short, l^=$N_1$, -] (0,2);
\end{circuitikz}
\end{center}
 \caption{A circuit with a battery, three resistors and a light bulb}
\end{figure}

This circuit has five branches, which are shown marked in color: a battery (in blue), three resistors ($R_1, R_2,$ and $R_3$, in red), and the light bulb (in green). The three nodes of the circuit ($N_1$, $N_2$, and $N_3$) connect the branches. Additionally, there are three loops, all of which have the same starting and ending point, $V_{battery}$. The first loop passes through $R_2$ and $R_1$, and returns to the starting point. Similarly, the second loop passes through $R_3$ and $R_1$, and returns. The third, and final, loop runs through the light bulb, then $R_1$, and returns. 

\subsection{Kirchhoff's Laws}
\textbf{Kirchhoff's Current Law (KCL)}
\\
Observe a circuit up until a node, past which the path of the circuit splits in two. The current encountering the node does not accumulate (as it would e.g. in a battery). Instead, all electrons flowing to that node split up between the available paths, and continue to flow through the circuit. This is Kirchhoff’s current law (KCL), which states that the algebraic sum of all currents in a node is equal to zero. 
\begin{align}
\sum_{j=1}^{N} i_{j}(t) = 0,
\end{align}
where $i_{j}(t)$ is the $j$'th current entering the node through branch $j$, with $N$ branches connected to the node. \cite[page~32]{bcircuit}
\\
\\
\textbf{Kirchoff's Voltage Law (KVL)}
\\
KVL states: In a circuit, the algebraic sum of all voltages in a loop is equal to zero. This can be expressed mathematically  as,
\begin{align}
\sum_{j=1}^{N} v_{j}(t) = 0,
\end{align}
where $v_{j}(t)$ is the voltage in the $j$'th voltage with $N$ voltages.\citep[page~34]{bcircuit}\\

