\chapter{RC Circuits} \label{chap:RC}
%This project will focus on one type of circuit, RC circuits.
%\\
%Behøver det ovenstående at være der?
An RC circuit is an electrical circuit consisting of resistors and capacitors. In its simplest form, it is composed of one of each. In that case a capacitor is placed in series with a resistor and a voltage input. When a voltage input is applied the capacitor charges, when the voltage input switched off, the capacitor will begin to discharge. A circuit diagram of an RC circuit is shown below:
\\
\begin{figure}[H]
 \begin{center}
\begin{circuitikz}[american voltages]
\draw (0,0)
to[battery, battery1=$V$] (0,2)
to[short, -] (4,2)
to[C=$C$] (4,0)
to[resistor, R=$R$] (0,0);
\end{circuitikz}
\end{center}

 \caption{A simple RC circuit with a voltage input, a capacitor and a resistor.}
\end{figure}
\section{Transient analysis}
\label{sec371}
The charging and discharging of a capacitor can be described mathematically. The equations for charging and discharging can be derived mathematically, using transient analysis.
\\
\\
\noindent\textbf{Discharging of a capacitor}\\
In the case of an RC circuit, KCL in section \ref{Klaws} can be rewritten as the following equation:
\begin{align}
i_{C}(t)+i_{R}(t)&=0, \nonumber \\
i_{C}(t)&= -i_{R}(t), \label{Ic-Ir}
\end{align}
where $i_C(t)$ is the current through the capacitor, and $i_R(t)$ is the current through the resistor. From \eqref{iC}, the current through the capacitor is:
\begin{align}
	i_C(t) = C\frac{dv_C(t)}{dt}.\label{iC=Cdv}
\end{align}
The equation for the current through the resistor \eqref{Ohm} can be rewritten as:
\begin{align}
	i_R(t) = \frac{v_R(t)}{R}. \label{iR=vR}
\end{align}
Inserting \eqref{iC=Cdv} and \eqref{iR=vR} into \eqref{Ic-Ir} yields:
\begin{align*}
	C\frac{dv_C(t)}{dt} = \frac{-v_R(t)}{R}.
\end{align*}
Both sides are divided by $C$:
\begin{align}
	\frac{dv_C(t)}{dt} =-v_R(t) \frac{1}{RC}.
	\label{eq:dvC(t)}
\end{align}
From KVL, it can be derived that:
\begin{align}
	v_C(t)- v_R(t) = 0, \nonumber\\
	v_R(t) = v_C(t). \label{vR(t)}
\end{align}
Equation \eqref{vR(t)} is inserted into \eqref{eq:dvC(t)}:
\begin{align*}
	\dfrac{dv_C(t)}{dt} &= \dfrac{-1}{RC}v_C(t).
\end{align*}
The equation is now in the same form as \eqref{SDEG}, and can be solved as follows:
\begin{align}
\int \dfrac{1}{v_C(t)}dv_C(t) =& \dfrac{-1}{RC} \int dt, \nonumber \\
\ln\big(v_C(t)\big) =& \dfrac{-t}{RC} + A, \nonumber\\
v_C(t) =& e^{\frac{-t}{RC}}e^{A}.\label{V_eA}
\end{align}
Since $A$ is the constant of integration, $e^A$ is constant as well.
\\
By definition, the initial voltage of a fully charged capacitor is $v_C(0)=v_{input}$:
 \begin{align*}
	v_C(0) &= e^{\frac{0}{RC}}e^A, \\
	v_C(0) &= e^A.
 \end{align*}
Therefore $e^A = v_{input}$, and this can be inserted into \eqref{V_eA}. Furthermore $RC$ is replaced with $\tau$:
\begin{align}
\label{V_down}
\Aboxed{
 v_{discharge}(t) = v_{input}e^{\frac{-t}{\tau}}
 }
\end{align}
The function in \eqref{V_down} describes how the voltage decreases over time, when the capacitor is discharged.
\\
\\
\textbf{Charging of a capacitor}\\
In the case of an RC circuit where the capacitor gets charged, KVL in section \ref{Klaws} can be rewritten as the following equation:
\begin{align}
v_{input}-v_R(t)-v_C(t) =& 0, \nonumber \\
v_{input} =& v_R(t)+v_C(t), \label{V_B}
\end{align}
where $v_{input}$ is the voltage of the input source, $v_R(t)$ is the voltage across the resistor, and $v_C(t)$ is the voltage across the capacitor. 
\\
From \eqref{Ohm}, the voltage across the resistor can be expressed as $v_R(t)=i_{R}(t) R$. From \eqref{QCV} the voltage across the capacitor can be expressed as $v_C(t)=\dfrac{q_C (t)}{C}$. These voltage definitions are inserted into \eqref{V_B}:
\begin{align}
v_{input} =& i_{R}(t) R + \dfrac{q_C (t)}{C}. \label{Vb=IR}
\end{align}
As stated in \eqref{I=dq/dt}, current is defined as $i(t) =\dfrac{dq(t)}{dt}$, and this is inserted into \eqref{Vb=IR}:
 \begin{align*}
 	v_{input} &= \frac{dq_R(t)}{dt} R + \frac{q_C (t)}{C}.
 \end{align*}
Both sides are divided by $R$:
\begin{align}
\dfrac{v_{input}}{R} &= \dfrac{dq_R(t)}{dt} + \dfrac{1}{RC}q_C(t).\label{Vb/R} 
\end{align}
This is now a linear differential equation of the first order in the form of \eqref{FODE_form}:
\begin{align*}
\dfrac{dy}{dx}+h(x)y=g(x).
\end{align*}
This can be solved with the general solution in \cref{linethe}:
\begin{align*}
y&=e^{-H(x)}\left(\int e^{H(x)}g(x)dx+C\right).
\end{align*}
The terms in \eqref{Vb/R} represent these terms in the general solution:
\begin{align*}
y &= q_C(t),
\\
h(x) &= \dfrac{1}{RC},
\\
H(x) &= \int \dfrac{1}{RC}dt=\dfrac{t}{RC},
\\
g(x) &= \dfrac{v_{input}(t)}{R},
\\
\dfrac{dy}{dx} &= \dfrac{dq_R(t)}{dt}.
\end{align*}
Then the general solution for \eqref{Vb/R} is:
\begin{align*}
q_C(t) &= e^{\frac{-t}{RC}}\int_{0}^{t}e^{\frac{t}{RC}}\dfrac{v_{input}}{R}dt.
\end{align*}
The constant is placed outside of the integral:
\begin{align}
q_C(t) &= e^{\frac{-t}{RC}}\dfrac{v_{input}}{R}\bigg(\int_{0}^{t}e^{\frac{t}{RC}}dt+C \bigg). \label{Q_1}
\end{align}
The integral is now solved by substitution. First $u$ and $\dfrac{du}{dt}$ are defined:
\begin{align*}
u &= \dfrac{t}{RC},
\\
\dfrac{du}{dt}&=\dfrac{1}{RC},
\\
dt &=RC du.
\end{align*} 
By inserting these definitions into \eqref{Q_1}, the equation looks as follows
\begin{align*}
q_C(t) &= e^{\frac{-t}{RC}}  \dfrac{v_{input}}{R} \int_{0}^{\frac{t}{RC}} R  C  e^u du,
\\
q_C(t) &= e^{\frac{-t}{RC}} v_{input} C \left[e^u\right]_{0}^{\frac{t}{RC}},
\\
q_C(t) &= v_{input}  C e^{\frac{-t}{RC}}\left(e^{\frac{t}{RC}}-1\right),
\\
q_C(t) &= v_{input}  C  \left(1-e^{\frac{-t}{RC}}\right).
\end{align*} 
The voltage across a capacitor is given as $v_C(t)=\dfrac{q_C(t)}{C}$. By dividing the equation above with $C$, and replacing $RC$ with $\tau$, the function for charging a capacitor is found:
\begin{align}
\label{V_up}
\Aboxed{
v_{charge}(t)=v_{input}\left(1-e^{\frac{-t}{\tau}}\right)
}
\end{align}
\section{RC circuit experiment} \label{c: RC_exp}
In the following section an RC circuit experiment has been conducted. A step voltage input applies current to the circuit until the capacitor is charged. The current is then switched off, until the capacitor is discharged. The voltage across the capacitor will be measured for the charge and discharge and will be compared to \eqref{V_down} and \eqref{V_up}.

\subsection{Experiment set-up}
This experiment was conducted by creating an RC circuit, which contained a resistor, capacitor, and a voltage source, with the following values:
\begin{align*}
 R =& 4770\Omega, \\
 C =& 97.61nF,
 \\
 v_{input} =& 1 V.
\end{align*}
The circuit was set up as follows:
\begin{figure}[H]
	\begin{center}
\begin{circuitikz}[american voltages]
\draw (0,0)
to[sqV, sqV=$1V$] (0,2)
to[short, -] (4,2)
to[C=$97.61nF$] (4,0)
to[resistor, R=$4770 \Omega$] (0,0);
\end{circuitikz}
\end{center}
	\caption{An RC circuit with a resistor (4770$\Omega$), capacitor (97.61nF). and voltage input (1V).}
\end{figure}
\noindent
The RC circuit was constructed using the Analog Discovery 2. A representation of this can be seen in figure \ref{rc_flow}.
\begin{figure}[H]
	\center
		\includegraphics{fig/img/charging.pdf}
		%[clip, trim=0cm 18cm 0cm 0cm, scale=0.6]
	\caption{Setup of the RC circuit used in the experiment.}
	\label{rc_flow}
\end{figure}
\noindent
The first part of the set-up is inserting a positive waveform generator (yellow) and a positive scope (blue) showing the waveforms. The waveform generator uses an AC step voltage. The current then passes through a resistor ($4.77 k\Omega$) and a capacitor ($97.61 nF$). Another scope is inserted to measure the charging and discharging of the capacitor. First, the current passes through a positive scope, then the capacitor, and back through a negative scope. Lastly, the current passes through a negative scope and into ground.
\subsection{Theoretical values}
In this experiment a step voltage input was used, for which the frequency input has to be calculated. To determine this frequency one has to decide for how long the capacitor should charge and discharge. To do this, a table with percentual voltage of the capacitor charges and discharges is made with \eqref{V_down} and \eqref{V_up}.
\\
After $1\tau$ the percentual voltage of the capacitor when charging can be calculated as follows:
\begin{align*}
v_{charge}(1\tau)=v_{input}\bigg(1-e^{\frac{-\tau}{\tau}}\bigg )=0.632v_{input}
\end{align*}
\\
After $1\tau$ the percentual voltage of the capacitor when discharging can be calculated as follows:
\begin{align*}
v_{discharge}(1\tau)=v_{input}e^{\frac{-\tau}{\tau}}=0.368v_{input}
\end{align*}
The following table shows how much the capacitor has charged, or discharged, as a function of time:
\begin{table}[H]
\center
\begin{tabular}{|l|l|l|l|l|l|l|l|}
\hline
                & $1\tau$  & $2\tau$  & $3\tau$  & $4\tau$  & $5\tau$  & $\cdots$ & $10\tau$      \\ \hline
$v_{charge}$    & $63.2\%$ & $86.5\%$ & $95.0\%$ & $98.2\%$ & $99.3\%$ & $\cdots$ & $ \sim 100\%$ \\ \hline
$v_{discharge}$ & $36.8\%$ & $13.5\%$ & $5.0\%$  & $1.8\%$  & $0.7\%$  & $\cdots$ & $\sim 0\%$    \\ \hline
\end{tabular}
\caption{The calculated percentual change in the voltage across the capacitor when it is charged and discharged at times from $1\tau$ to $10\tau$.}
\end{table}
\noindent
%In this experiment firstly the capacitor charges for $10\tau$, where it is considered fully charged. The capacitor is then discharged for $10\tau$, where it is considered fully discharged. The total time the capacitor is charged and then discharged is then: rettet til
The time it takes the capacitor to fully charge ($10\tau$) and discharge ($10\tau$) is used to define the period of the input signal:
\begin{align}
2\cdot 10\tau=2\cdot 10 \cdot RC&=2\cdot 10 \cdot 4770\Omega \cdot 97.61 nF, \nonumber
\\
&=9.32 ms. \label{rc:tau:full}
\end{align} 
The frequency of the input signal is then set as the inverse of this period. Meaning the frequency of the input signal is:  $(9.32ms)^{-1} = 107.3 Hz$.
\\
The data from the theoretical values and the test results can now be compared using python.
\subsection{Data comparison}
\begin{figure}[H]
\center
\includegraphics[scale=0.6]{fig/img/eks_1}
\caption{The measured and simulated data for the charge and discharge of a capacitor.}
\label{fig:Cap}
\end{figure}
\noindent
As shown on figure \ref{fig:Cap}, the test results are very close to the calculated theoretical values. The coefficient of determination, $R^2$, is almost equal to 1, which means that the two plots are almost identical. If the measured data was exactly the same as the simulated data, $R^2$ would be $1$. Since the model only considered the charging and discharging of the capacitor in an ideal world, there is some uncertainty. This uncertainty could stem from resistance in the wires, which was not accounted for in the simulation, or other unknown sources of error.