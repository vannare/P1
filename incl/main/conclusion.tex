\chapter{Conclusion}
In this project, the theory behind RC circuits has been clarified. Furthermore, it is found that when charging or discharging a capacitor, both basic circuit theory and differential equations have to be taken into account in order to explain it. By using Kirchhoff's laws, a differential equation can be established, as found in chapter \ref{ch:basic}. When solving this equation, the equations for charging \eqref{V_up} and discharging \eqref{V_down} of a capacitor were found. To test the validity of the two equations, a circuit was constructed using the theory from chapter \ref{ch:basic}, and a simulation was used to validate these equations. The results from the experiment validated the simulated charging and discharging curves of the capacitor to a high degree of certainty, as seen in \ref{fig:Cap}. Therefore, the theory behind the experiment can be considered legitimate.
\\ \\
Aside from the charging and discharging of a capacitor, an experiment using high- and low-pass filters was conducted. Complex numbers and the Laplace transform are introduced to find the most essential equations in relation to high- and low-pass filters. Moreover, differential equations and circuit analysis had to be taken into account. Kirchhoff's laws were used to construct a differential equation, which was then Laplace transformed from functions of time into complex functions dependent on frequencies. This equation led to the transfer function for high- and low-pass filters, from where the cut-off frequency was found. As seen in figures \ref{fig:lpf:exp} and \ref{fig:hpf:exp}, the simulated and experimental values are almost identical, which confirms the validity of the equations derived in chapter \ref{chap:high_low}.
\\ \\
Ultimately, the project has studied different aspects of RC circuits, which have been tested to be true. 
