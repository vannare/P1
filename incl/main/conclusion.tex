\chapter{Conclusion}
In this project, the theory connected with an RC circuit has been clarified. Furthermore, it is found that when charging or discharging a capacitor, both basic circuit theory and differential equations have to be taken into account (to explain it). As found in chapter \ref{ch:basic}, by using Kirchhoff's laws, a differential equation can be established. When solving this equation, the equation for charging \eqref{V_up} and discharging \eqref{V_down} of a capacitor was found. To test the validity of the two equations, a circuit was constructed using the theory from chapter \ref{ch:basic} and a simulation was used to validate these equations. By using the simulations it was found that when plotting the equations, it was so close to the theoretical values that they can be considered identical as shown on figure \ref{fig:Cap}. Since the found equations confirm the theoretical values from the simulation, the theory can be considered legitimate. 
\\ \\
Aside from the charging and discharging of a capacitor, an experiment using a high- and low-pass filter. Complex numbers and the Laplace transform is introduced to find the most essential equation in relation to high- and low-pass filters. Moreover, differential equations and circuit analysis still had to be taken into account. Kirchhoff's laws were again used to construct a differential equation, which was then Laplace transformed from functions of time into complex functions dependant on frequencies. This equation led to the transfer function for high- and low-pass filters from where the cut-off frequency was found. This was then simulated to see if the theoretical values were identical to the found values. This can be seen on figure \ref{fig:lpf:exp} and \ref{fig:hpf:exp}, where it can be seen that these are almost identical, and the found equations can then be confirmed to be true. 
\\ \\
Ultimately, the project has studied different aspects of RC circuits, which have been tested to be true. 
