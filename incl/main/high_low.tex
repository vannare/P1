\chapter{High-pass and low-pass filters}
The main purpose of filters is to sort out unwanted frequencies from different wave-types. Filters are often used on sound files, to control the bass or the high pitch noises. There are different kinds of filters that can sort out unwanted frequencies. In this section the focus is on high and low-pass filters.

\section{Low-pass filters}
The purpose of a low-pass filter is to filter out unwanted high frequencies. Listening to a such tone brings out the bass and sorts out higher pitches. The low-pass filter is in other words allowing frequencies from 0$Hz$ to a chosen cut-off value. This cut-off value can be calculated using the following equation:
\begin{align*}
f_{cutoff}=\dfrac{1}{2 \pi RC}
\end{align*}
This cut-off value describes a point on the bode plot graph, where the amplitude is decreased by $29.3\%$ and the output is decreased by 3$dB$. The cut-off point is also the point, where the filtering starts getting efficient. This can be expressed using a bode plot, with the frequency on the $x-axis$ and the decibel on the $y-axis$. The bode plot looks like this: \\
PLOT \\
The $x-axis$ is a logarithmic axis, where the output is decreased by 20 decibel per decade after the cut-off point.\
The difference between a high-pass filter and a low-pass can be illustrated using a circuit diagram:
\begin{figure}[H]
	\begin{center}
\begin{circuitikz}[american voltages]
\draw (0,0)
to[sqV, sqV=$V_{AC}$] (0,2)
to (6,2)
to[short, -] (4,2)
to[C=$C$] (4,0)
to (6,0)
to (4,0)
to [resistor, R=$R$] (0,0);
\draw [>=latex', <->] (6,1.75) -- node[anchor=west] {$V_{output}$} (6,0.25);
\end{circuitikz}
\end{center}

\end{figure}
In the illustration above the voltage output is measured around the capacitor, which decreases the high frequencies and leaves the low frequencies unchanged. The output wave before the cut-off point will stay almost unchanged, and after the cut-off point, the amplitude is decreasing and approaching a graph that looks like a DC current. 


- bode plot \\
- circuit \\
- input/output \\
- short example (eg. we want a cut-off at 150Hz)