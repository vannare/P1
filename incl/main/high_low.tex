\chapter{High-pass and low-pass filters}
The main purpose of filters is to sort out unwanted frequencies from different wave-types. 

\section{Low-pass filters}
The purpose of a low-pass filter is to filter out unwanted high frequencies. Listening to a such tone brings out the bass and sorts out higher pitches. The low-pass filter is in other words allowing frequencies from 0$Hz$ to a chosen cut-off value. This cut-off value can be calculated using the following equation:
\begin{align*}
f_{cutoff}=\dfrac{1}{2 \pi RC}
\end{align*}
This cut-off value describes a point on the bode plot graph, where the amplitude is decreased by $29.3\%$ and the output is decreased by 3$dB$. The cut-off point is also the point, where the filtering starts getting efficient. This can be expressed using a bode plot, with the frequency on the $x-axis$ and the decibel on the $y-axis$. The bode plot looks like this: \\
PLOT \\
The $x-axis$ is a logarithmic axis, where the output is decreased by 20 decibel per decade after the . 
\begin{figure}[H]
	\begin{center}
\begin{circuitikz}[american voltages]
\draw (0,0)
to[sV, sV=$v_{input}(t)$] (0,2)
to (6,2)
to[short, -] (4,2)
to[C=$C$] (4,0)
to (6,0)
to (4,0)
to [resistor, R=$R$] (0,0);
\draw [>=latex', <->] (6,1.75) -- node[anchor=west] {$v_{output}(t)$} (6,0.25);
\end{circuitikz}
\end{center}

\end{figure}


- bode plot \\
- circuit \\
- input/output \\
- short example (eg. we want a cut-off at 150Hz)