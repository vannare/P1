\lstlistingname{ Data handling}
%\lstset{caption={generic_plot.py}}
\begin{lstlisting}[breaklines]
import matplotlib.pyplot as plt
import numpy as np


# constants of circuit, w is angular frequency inputs
w = np.linspace(0, 10**7, 10**5)
R = 1000
C = 97.61*10**-9


# function for calculating the gain of an LPF
def trans_func(wf):
    return 20*np.log10(1/np.sqrt(1+(R*C*wf)**2))


# the out gain
out = trans_func(w)

# cutoff frequency values
fc_x = 1/(R*C)
fc_y = trans_func(fc_x)

# plotting and prettyfying
plt.xscale('log')
plt.plot(w, out, 'black')
plt.grid('true')
plt.xlabel(r'The angular frequency of the input signal $\left[\dfrac{rad}{s}\right]$')
plt.ylabel('Gain in decibels [dB]')
plt.title('Example of a bode plot')
plt.scatter(fc_x, fc_y, edgecolor = 'blue')
plt.annotate(r'$f_c$', (fc_x+4000, fc_y), size = 14)
plt.tight_layout()
plt.ylim(-59, 2)

# cutting up the intervals for coloring
blue_reg_x = w[w <= fc_x]
red_reg_x = w[w >= fc_x]
blue_reg_y = trans_func(blue_reg_x)
red_reg_y = trans_func(red_reg_x)

# changing the first and last points to color in all of the plot
blue_reg_y[0] = out[-1]
blue_reg_y[-1] = out[-1]
red_reg_y[0] = out[-1]
red_reg_y[-1] = out[-1]

# coloring the plot
plt.fill(blue_reg_x, blue_reg_y, 'blue', alpha = 0.3)
plt.fill(red_reg_x, red_reg_y, 'red', alpha = 0.3)

plt.savefig('bode_example.pdf')

\end{lstlisting}